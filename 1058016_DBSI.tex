\documentclass{article}
\usepackage[left=-1cm, right=2cm]{geometry}
\usepackage{amsmath,amssymb}
\usepackage{amsmath, amsfonts, amssymb}
\usepackage{wasysym}
\usepackage{multicol}
\usepackage{graphicx}
\usepackage{latexsym}
\usepackage{xmpmulti}
\usepackage{pgfpages}
\usepackage{times}
\usepackage{fullpage}
\usepackage{enumitem}
\usepackage{float}
\usepackage{tikz}
\usetikzlibrary{automata}
\usepackage{graphicx}
\usepackage{caption}
\usepackage{subcaption}
\usepackage{adjustbox}	
\usepackage{tikz}
\usepackage{enumitem}
\usepackage{lipsum}
\newcommand{\R}{\mathbb{R}}
\newcommand{\Q}{\mathbb{Q}}
\newcommand{\Z}{\mathbb{Z}}
\newcommand{\N}{\mathbb{N}}
\newcommand{\C}{\mathbb{C}}
\newcommand{\B}{\mathbb{B}}
\usepackage[english]{babel}
\usepackage{amsthm}
\usepackage{blindtext}
\usepackage{hyperref}
\usepackage{algorithm}
\usepackage{algpseudocode}


\newtheorem{definition}{Definition}
\newtheorem{theorem}{Theorem}
\newtheorem{corollary}{Corollary}
\newtheorem{lemma}[theorem]{Lemma}
\newtheorem{example}[theorem]{Example}

% "draft" enables (manual) caching of figures for faster builds

\begin{document}
Departmental Coversheet

Hillary term 2022
Mini-project

Paper title: Database System Implementation

Candidate Number: 1058016

Your degree: MSc Advanced Computer Science

\newpage

\begin{enumerate}
	\item 
	\begin{enumerate}
	\item 
	The data structure in this paper is a six-column triple table. The subject, predicate, and object of a triple are encoded as integers in the first three columns, $R_s$, $R_p$, and $R_o$, respectively. Conceptually, there are three linked lists, an $sp$-list that connects all triples with the same $R_s$ grouped by $R_p$, an $op$-list that associates all triple with the same $R_o$ grouped by $R_p$, and a $p$-list that relates all triples with the same $R_p$ without any grouping. In the table, the last three columns, $N_{sp}$, $N_{op}$, and $N_p$, store the next-pointes, which are going to be the row number in the triple table in the actual implementation.
	
	Six index maps are also maintained. $I_s$, $I_p$, and $I_o$ store the head of the $sp$-list, $p$-list, and $op$-list, respectively. $I_{sp}$ maps $s$ and $p$ to the first occurrence of the triple with the same $s$ and $p$ in the $sp$-list. So does $I_{op}$. $I_{spo}$ stores the row number of each triple in the table. 
	
	\item {\large A}DD(Triple $t$) consists of two parts; first, append a new row to the end of the RDF-index table, and second, associates the new row with all six index maps and alter the pointer columns, $N_{sp}$, $N_{op}$, and $N_p$, to point to the correct row.
	
	Since there is no need to worry about the concurrency in our setting, the RDF-index table is maintained as a fixed size vector of int list of size 6 with each position stands for different columns. Therefore, for each time we add a triple into the table, we simply append it to the last. However, if the triple is already in the $I_{spo}$ map, we skip it. When the size reaches its limit, the entire table will resize. 

\begin{algorithm}[H]
\caption{ADD (t)}\label{alg:cap}
\begin{algorithmic}

\If{$t$ in $I_{spo}$} \Comment{$t = (s, p, o)$ is a triple.}
\State \Return
\EndIf

\State $i$ = \# of elements in the triple-table

\If{$i + 1 > $ the size of the triple-table}
\State resize the triple table
\EndIf
\State $T_{new}$ = $[t.s, t.p, t.o, -1, -1, -1]$ \Comment{The last three columns are left for update later.}
\State triple-table[i] = $T_{new}$
\State $I_{spo}[t] = i$
\State Update remaining indexes
\end{algorithmic}
\end{algorithm}

The columns, $N_{sp}$, $N_{op}$, and $N_p$, are maintained in a linked-list-like manner and are updated simultaneously with the index maps. Take $N_{op}$ for example, if $T_{new}$ does not appear in the $I_o$ and $I_{op}$, it means that  $T_{new}$ is a new triple that the triple table never see before. Therefore, we insert $T_{new}$ into the index map $I_o$ and $I_{op}$ (case1). If we found a $T$ with $T.o = T_{new}.o$ and $T.o = T_{new}.p$, then we insert $T_{new}$ after $T$, and point the next of $T_{new}$ to the original next of the $T$ (case3). A special case is that when there is no next for $T$ and it is handled in case 2 of the Algorithm \eqref{alg:updateIop}. The case of $M_{sp}$ and $N_p$ is similar.

\begin{algorithm}[H]
\caption{Update $I_{op}$($T_{new}$)}\label{alg:updateIop}
\begin{algorithmic}
\State $T$ = the first triple with $T.o = T_{new}.o$ and $T.o = T_{new}.p$
\If{$T$ does not exist}\Comment{Case 1}
\State make $T_{new}$ the head of $I_o$ and $I_{op}$ 
\EndIf

\If{$T$ does not have $T_{next}$}\Comment{Case 2}
\State make $T_{new}$ the head of $I_o$ and $I_{op}$
\State $T_{new}.N_{op} = T$
\EndIf

\If{$T$ has $T_{next}$} \Comment{Case 3}
\State $T_{next} = T.N_{op}$
\State $T_{new}.N_{op} = T_{next}$
\EndIf

\end{algorithmic}
\end{algorithm}

There 



\end{enumerate}
\item 
\begin{enumerate}
\item RDF indexing data structure that implements Add and Evaluate functions.

This component is included in RDF\_index.cpp, in which ADD and EVALUATE are implemented as suggested in the problem. There are a few things that is slightly different from the paper. 

1) I used XXHASH\footnote{https://cyan4973.github.io/xxHash/} by Facebook instead of Jenkings hashing, because it achieves state-of-the-art excellent performance on both long and small inputs. 

2) Instead of open addressing, I eventually choose std::unordered\_map for index maps $I_{sp}$, $I_{op}$ and $I_{spo}$. I had an open addressing hash implemented in HashTable.cpp and HashTable.h (attached in the submission), but it was much slower than the unordered\_map so I changed it after one update. 

3) To match the patterns like $\langle X, Y, Z\rangle$, the paper suggests to iterate over the triple table; if we want $X = Y$,  we skip those $X\neq Y$. I modify this a little bit to improve the efficiency. Again, $\langle X, X, Z\rangle$, for example, we first iterate $I_s$, and for each $s$ in $I_s$, we find if $I_{sp}$ includes hash($s, s$), if yes, we traverse over the triple table. As shown in the pseudo-code in Algorithm \eqref{alg:evaluateXXP}.

\begin{algorithm}[H]
\caption{Evaluate $\langle X, Y, Z\rangle$}\label{alg:evaluateXXP}
\begin{algorithmic}
\For{$s$ in $I_s$}
\State $i = $ hash($s, s$)
\If{$i$ in $I_{sp}$}	
\State Evaluate\_SPZ($s, s$) \Comment{Evaluate\_SPZ }
\EndIf
\EndFor
\end{algorithmic}
\end{algorithm}

$\langle X, Y, Y\rangle$ and $\langle X, Y, X\rangle$ are similar. For $\langle X, X, X\rangle$, we only need to traverse  $s$ in $I_s$ and find if $\langle s, s, s\rangle$ is in $I_{spo}$ as shown in  Algorithm \eqref{alg:evaluateXXX}.

\begin{algorithm}[H]
\caption{Evaluate $\langle X, X, X\rangle$}\label{alg:evaluateXXX}
\begin{algorithmic}
\For{$s$ in $I_s$}
\State Evaluate\_SPO($s, s, s$) 
\EndFor
\end{algorithmic}
\end{algorithm}

\item The engine for evaluating BGP SPARQL queries.

This part is in the files SPARQL\_engine.cpp. And is implemented strictly based on the model answer. A few additional lines are added for printing or improving performance.

\item The greedy join order optimization query planner.

I see what the model answer is trying to say but I believe there are some minor errors so I change it slightly but it is still $O(n)$, where $n$ is the number of triple patterns we try to fit. 


\item The component for parsing and importing Turtle files.

This part is included in query\_parser.cpp. My parser not only read the query but also check if the triple patterns are correct such as if the query is asking for something that is not presented in the database, it will throw a warning right away instead of feeding it into the query engine to create overhead. 

\item The parser for SPARQL queries

This part is included in interface.cpp.

\item The component implementing the command line

This part is included in interface.cpp.

\end{enumerate}

\item[3.a]
\begin{enumerate}
\item 

Hardware and Software Configuration:

\begin{enumerate}
\item Model Identifier:	MacBookPro14,3

\item Processor Name:	Quad-Core Intel Core i7

\item Processor Speed:	2.9 GHz

\item Number of Processors:	1

\item Total Number of Cores:	4

\item L2 Cache (per Core):	256 KB

\item L3 Cache:	8 MB

\item Memory:	16 GB

\item Operating System: macOS Big Surf, Version 11.6 (20G165)

\item Compiler Version: Apple clang version 11.0.0 (clang-1100.0.33.8)

\end{enumerate}

\item Test Protocol:

\begin{enumerate}
\item Turn off all irrelevant applications.
\item Start a clean terminal.
\item make all and run the output file.
\item If it is measuring the time needed to load and index the
data, markdown the time and repeat this process for 10 times. Take off the highest and the lowest and find the average. 
\item To measure the time needed to evaluate the query, we choose COUNT instead of SELECT and for each query, we markdown the time and repeat this process for 10 times. Take off the highest and the lowest and find the average. 
\end{enumerate}

\end{enumerate}

\item[3.b] The time needed to load and index the data and the time needed to produce all query answers for each RDF graph and query are showed in Table 1 and Table 2, respectively.

\begin{table}[H]\centering
\begin{tabular}{|l|l|l|l|}
\hline
                   & LUBM-010-mat & LUBM-010-mat & LUBM-010-mat \\ \hline
Load \& Index Time & 1064         & 11879        & 122583       \\ \hline
\end{tabular}
\caption{Time needed to load and index the data in ms.}
\end{table}



\begin{table}[H]\centering
\begin{tabular}{|l|l|l|l|}
\hline
    & LUBM-001-mat & LUBM-010-mat & LUBM-100-mat \\ \hline
q1  & 8            & 95           & 817          \\ \hline
q2  & 22           & 393          & 25387        \\ \hline
q3  & 25           & 261          & 2533         \\ \hline
q4  & 2            & 26           & 211          \\ \hline
q5  & 37           & 349          & 3489         \\ \hline
q6  & 25           & 254          & 2376         \\ \hline
q7  & 133          & 1375         & 14250        \\ \hline
q8  & 128          & 1060         & 11367        \\ \hline
q9  & 250          & 2618         & 26922        \\ \hline
q10 & 29           & 328          & 3305         \\ \hline
q11 & 1            & 13           & 119          \\ \hline
q12 & 0            & 2            & 27           \\ \hline
q13 & 35           & 322          & 3485         \\ \hline
q14 & 18           & 194          & 1809         \\ \hline
\end{tabular}
\caption{Time needed to produce  all query answers in ms.}
\end{table}

\item[3.c]

\begin{enumerate}
\item There
\end{enumerate}

\end{enumerate}
















\end{document}